\begin{lemma}[Disjointness/subtyping exclusivity]
  \lbl{lem:disjointness-subtyping-exclusivity}
  ~\\\indent
  The type disjointness and subtyping relations are mutually exclusive.
\end{lemma}

\begin{proof}
  We prove mutual exclusivity of type disjointness and subtyping by first defining $\⟦\· \⟧_Γ$, a mapping from \SystemFm types (in a given context $\Γ$) into non-empty subsets of a newly defined set $\P$. We then show that the subtyping relation in \Fm corresponds to a subset relation in $\P$, and that the type disjointness relation in \Fm ($\disj$) corresponds to set disjointness relation in $\P$.

  This set-theoretical view of subtyping and disjointness renders the proof trivial.
  Indeed, suppose there exist two types $\S$ and $\T$ with $\Γ \⊢ \S \<: \T$ and $\Γ \⊢ \disj(\S, \T)$.
  $\⟦ \S \⟧_Γ$ and $\⟦ \T \⟧_Γ$ are two non-empty sets which are both intersecting and disjoint, a contradiction.

  We first define $\P$ as $\P \= \{ \Λ, \V \} \∪ \C$. Elements of $\P$ can be understood as equivalence classes for \SystemFm values: $\Λ$ corresponds to all abstraction values, $\V$ corresponds to type abstraction value, and elements of $\C$ correspond to their respective constructors.
  The definition of $\⟦\· \⟧_Γ$ makes this correspondence apparent.

  We define $\⟦\· \⟧_Γ$, a mapping of \SystemFm types into subsets of $\P$ in a given context $\Γ$:

  \begin{align*}
    \text{$\⟦ \Top \⟧_Γ$} &=
    \text{$\P$}\\
    %
    \text{$\⟦ \X \⟧_Γ$} &=
    \text{$\⟦ \T \⟧_Γ \quad$ where $\X \<: \T \∈ \Γ$}\\
    %
    \text{$\⟦ \T_{1} \→ \T_{2} \⟧_Γ$} &=
    \text{$\{ \Λ \}$}\\
    %
    \text{$\⟦ \∀ \X \<: \U_{1}.~ \T_{2} \⟧_Γ$} &=
    \text{$\{ \V \}$}\\
    %
    \text{$\⟦ \{ \new \C_{1} \} \⟧_Γ$} &=
    \text{$\{ \C_{1} \}$}\\
    %
    \text{$\⟦ \C_{1} \⟧_Γ$} &=
      \{ \c \∈ \C~|~(\c, \C_1) \∈ \Ψ \}\\
    \⟦ \T_s \match \{ \S_i \⇒ \T_i \} \otherwise \T_d \⟧_\Γ &=
      \begin{cases}
        \text{$\⟦ \T_n \⟧_Γ$\phantom{\P} if $\⟦ \T_s \⟧_Γ \⊂ \⟦ \S_n \⟧_Γ$}\\
        \phantom{\text{$\⟦ \T_n \⟧_Γ$}}\phantom{\P}\text{ and $\∀ \m<n.~ \⟦ \T_s \⟧_Γ \∩ \⟦ \S_m \⟧_Γ \= \{ \}$}\\
        \text{$\⟦ \T_d \⟧_Γ$\phantom{\P} if $\∀ \m.~ \⟦ \T_s \⟧_Γ \∩ \⟦ \S_m \⟧_Γ \= \{ \}$}\\
        \P\phantom{\text{$\⟦ \T_n \⟧_Γ$}}\text{ otherwise}
      \end{cases}
  \end{align*}

  We show that the subtyping relation in \Fm corresponds to a subset relation in $\P$, and that the type disjointness relation in \Fm ($\disj$) corresponds to set disjointness in $\P$.
  In other words, we prove the follows statements:

  \begin{enumerate}
    \item $\Γ \⊢ \S \<: \T$ implies $\⟦ \S \⟧_Γ \⊂ \⟦ \T \⟧_Γ$,
    \item $\Γ \⊢ \disj(\S, \T)$ implies $\⟦ \S \⟧_Γ \∩ \⟦ \T \⟧_Γ \= \{ \}$,
  \end{enumerate}

  Both statements are proved simultaneously by induction on derivations of $\Γ \⊢ \disj(\V, \W)$ and $\Γ \⊢ \S \<: \T$.
  More precisely, the induction is done on the cumulative depth of both derivation tree.
  \begin{enumerate}
    \item ($\Γ \⊢ \S \<: \T$ implies $\⟦ \S \⟧_Γ \⊂ \⟦ \T \⟧_Γ$)
    \begin{itemize}
      \Case\SRefl:
      \quad $\T \= \S$

      $\⟦ \S \⟧_Γ \= \⟦ \T \⟧_Γ$ and the result is immediate.

      \Case\STrans:
      \quad $\Γ \⊢ \S \<: \U$
      \quad $\Γ \⊢ \U \<: \T$

      By the IH, $\⟦ \S \⟧_Γ \⊂ \⟦ \U \⟧_Γ$ and $\⟦ \U \⟧_Γ \⊂ \⟦ \T \⟧_Γ$. From subset transitivity, $\⟦ \S \⟧_Γ \⊂ \⟦ \T \⟧_Γ$, as required.

      \Case\STop:
      \quad $\T \= \Top$

      $\⟦ \Top \⟧_Γ \= \P$ coincides with the codomain of $\⟦\· \⟧_Γ$.
      Therefore, for any type $\T$, $\⟦ \T \⟧_Γ \⊂ \⟦ \Top \⟧_Γ$, as required.

      \Case\SSin:
      \quad $\S \= \{ \new \C_{1} \}$
      \quad $\T \= \C_{1}$

      $\⟦ \{ \new \C_{1} \} \⟧_Γ \= \{ \C_{1} \}$ and
      $\⟦ \C_{1} \⟧_Γ \= \{ \c \∈ \C~\|~\(\c, \C_{1}) \∈ \Ψ \}$.
      The result follows by reflexivity of $\Ψ$.

      \Case\STvar:
      \quad $\S \= \X$
      \quad $\X \<: \T \∈ \Γ$

      By definition $\⟦ \X \⟧_Γ \= \⟦ \T \⟧_Γ$, and the result is immediate.

      \Case\SArrow:
      \quad $\S \= \S_{1} \→ \S_{2}$
      \quad $\T \= \T_{1} \→ \T_{2}$
      \\\phantom{\textit{Case}~\SArrow:}
      \quad $\Γ \⊢ \T_{1} \<: \S_{1}$
      \quad $\Γ \⊢ \S_{2} \<: \T_{2}$

      $\⟦ \S_{1} \→ \S_{2} \⟧_Γ \= \⟦ \T_{1} \→ \T_{2} \⟧_Γ \= \{ \Λ \}$ and the result is immediate.

      \Case\SAll:
      \quad $\S \= \∀ \X \<: \U_{1}.~ \S_{2}$
      \quad $\T \= \∀ \X \<: \U_{1}.~ \T_{2}$
      \quad $\Γ, \X \<: \U_{1} \⊢ \S_{2} \<: \T_{2}$

      $\⟦ \∀ \X \<: \U_{1}.~ \S_{2} \⟧_Γ \= \⟦ \∀ \X \<: \U_{1}.~ \T_{2} \⟧_Γ \= \{ \V \}$ and the result is immediate.

      \Case\SPsi:
      \quad $\S \= \C_{1}$
      \quad $\T \= \C_{2}$
      \quad $\(\C_{1}, \C_{2}) \∈ \Ψ$

      $\⟦ \C_{1} \⟧_Γ \= \{ \c \∈ \C~\|~\(\c, \C_{1}) \∈ \Ψ \}$ and
      $\⟦ \C_{2} \⟧_Γ \= \{ \c \∈ \C~\|~\(\c, \C_{2}) \∈ \Ψ \}$.
      By transitivity of $\Ψ$, $\⟦ \C_{1} \⟧_Γ \⊂ \⟦ \C_{2} \⟧_Γ$, as required.

      \Case\SMatch1/2:
      \quad $\T_{1} \= \T_n$
      \quad $\T_{2} \= \T_s \match \{ \S_i \⇒ \T_i \} \otherwise \T_d$
      \\\phantom{\textit{Case}~\SMatch1/2:}
      \quad $\Γ \⊢ \T_s \<: \S_n$
      \quad $\∀ \m<n.~ \Γ \⊢ \disj(\T_s, \S_m)$

      \hfuzz=1.5pt
      Using the IH we get $\⟦ \T_s \⟧_Γ \⊂ \⟦ \S_n \⟧_Γ$, $\∀ \m<n.~ \⟦ \T_s \⟧_Γ \∩ \⟦ \S_m \⟧_Γ \= \{ \}$ and $\⟦ \T_s \match \{ \S_i \⇒ \T_i \} \otherwise \T_d \⟧_Γ \= \⟦ \T_n \⟧_Γ$.
      This last equality implies both $\⟦ \T_{1} \⟧_Γ \⊂ \⟦ \T_{2} \⟧_Γ$ and $\⟦ \T_{2} \⟧_Γ \⊂ \⟦ \T_{1} \⟧_Γ$, as required.

      \hfuzz=0pt
      \Case\SMatch3/4:
      \quad $\T_{1} \= \T_d$
      \quad $\T_{2} \= \T_s \match \{ \S_i \⇒ \T_i \} \otherwise \T_d$
      \\\phantom{\textit{Case}~\SMatch3/4:}
      \quad $\∀ \n.~ \Γ \⊢ \disj(\T_s, \S_n)$

      By the IH $\∀ \n.~ \⟦ \T_s \⟧_Γ \∩ \⟦ \S_n \⟧_Γ \= \{ \}$.
      By definition, $\⟦ \T_s \match \{ \S_i \⇒ \T_i \} \otherwise \T_d \⟧_Γ \= \⟦ \T_d \⟧_Γ$.
      This equality implies both $\⟦ \T_{1} \⟧_Γ \⊂ \⟦ \T_{2} \⟧_Γ$ and $\⟦ \T_{2} \⟧_Γ \⊂ \⟦ \T_{1} \⟧_Γ$, the desired result for \SMatch3 and \SMatch4 respectively.

      \Case\SMatch5:
      \quad $\S \= \S_s \match \{ \U_i \⇒ \S_i \} \otherwise \S_d$
      \quad $\T \= \T_s \match \{ \U_i \⇒ \T_i \} \otherwise \T_d$
      \\\phantom{\textit{Case}~\SMatch5:}
      \quad $\Γ \⊢ \S_s \<: \T_s$
      \quad $\∀ \n.~ \Γ \⊢ \S_n \<: \T_n$
      \quad $\Γ \⊢ \S_d \<: \T_d$

      By case analysis on $\⟦ \T \⟧_Γ$:
      \begin{enumerate}
        \item $\⟦ \T \⟧_Γ \= \⟦ \T_n \⟧_Γ$
        and $\⟦ \T_s \⟧_Γ \⊂ \⟦ \U_n \⟧_Γ$
        and $\∀ \m<n.~ \⟦ \T_s \⟧_Γ \∩ \⟦ \U_m \⟧_Γ \= \{ \}$:

        The IH gives $\⟦ \S_s \⟧_Γ \⊂ \⟦ \T_s \⟧_Γ$.
        Using $\∀ \m<n.~ \⟦ \T_s \⟧_Γ \∩ \⟦ \U_m \⟧_Γ \= \{ \}$, we get $\∀ \m<n.~ \⟦ \S_s \⟧_Γ \∩ \⟦ \U_m \⟧_Γ \= \{ \}$.
        Using $\⟦ \T_s \⟧_Γ \⊂ \⟦ \U_n \⟧_Γ$, we get $\⟦ \S_s \⟧_Γ \⊂ \⟦ \U_n \⟧_Γ$ (by subset transitivity).
        Therefore, $\⟦ \S \⟧_Γ \= \⟦ \S_n \⟧_Γ$, and the result follows from the IH.

        \item $\⟦ \T \⟧_Γ \= \⟦ \T_d \⟧_Γ$ and
        $\∀ \n.~ \⟦ \T_s \⟧_Γ \∩ \⟦ \U_n \⟧_Γ \= \{ \}$:

        The IH gives $\⟦ \S_s \⟧_Γ \⊂ \⟦ \T_s \⟧_Γ$.
        Using $\∀ \n.~ \⟦ \T_s \⟧_Γ \∩ \⟦ \U_n \⟧_Γ \= \{ \}$ we get $\∀ \n.~ \⟦ \S_s \⟧_Γ \∩ \⟦ \U_n \⟧_Γ \= \{ \}$.
        Therefore, $\⟦ \S \⟧_Γ \= \⟦ \S_d \⟧_Γ$, and the result follows from the IH.

        \item $\⟦ \T \⟧_Γ \= \P$:

        $\P$ is the codomain of $\⟦\· \⟧_Γ$, therefore $\⟦ \S \⟧_Γ \⊂ \⟦ \T \⟧_Γ$, as required.
      \end{enumerate}
    \end{itemize}

    \item ($\Γ \⊢ \disj(\S, \T)$ implies $\⟦ \S \⟧_Γ \∩ \⟦ \T \⟧_Γ \= \{ \}$)
    \begin{itemize}
      \Case\DXi:
      \quad $\S \= \C_{1}$
      \quad $\T \= \C_{2}$
      \quad $\(\C_{1}, \C_{2}) \∈ \Ξ$

      $\⟦ \C_{1} \⟧_Γ \= \{ \c \∈ \C~\|~\(\c, \C_{1}) \∈ \Ψ \}$ and
      $\⟦ \C_{2} \⟧_Γ \= \{ \c \∈ \C~\|~\(\c, \C_{2}) \∈ \Ψ \}$.
      By definition of Ξ, there is no class $\c$ such that both $\(\c, \C_{1}) \∈ \Ψ$ and $\(\c, \C_{2}) \∈ \Ψ$, and $\⟦ \C_{1} \⟧_Γ \∩ \⟦ \C_{2} \⟧_Γ \= \{ \}$.

      \Case\DPsi:
      \quad $\S \= \{ \new \C_{1} \}$
      \quad $\T \= \C_{2}$
      \quad $\(\C_{1}, \C_{2}) \∉ \Ψ$

      $\⟦ \{ \new \C_{1} \} \⟧_Γ \= \C_{1}$ and
      $\⟦ \C_{2} \⟧_Γ \= \{ \c \∈ \C~\|~\(\c, \C_{2}) \∈ \Ψ \}$.
      Since $\(\C_{1}, \C_{2}) \∉ \Ψ$, $\C_{1} \∉ \⟦ \C_{2} \⟧_Γ$ and $\⟦ \C_{1} \⟧_Γ \∩ \⟦ \C_{2} \⟧_Γ \= \{ \}$.

      \Case\DSub:
      \quad $\Γ \⊢ \S \<: \U$
      \quad $\Γ \⊢ \disj(\U, \T)$

      By the IH, $\⟦ \S \⟧_Γ \⊂ \⟦ \U \⟧_Γ$ and $\⟦ \U \⟧_Γ \∩ \⟦ \T \⟧_Γ \= \{ \}$.
      $\⟦ \S \⟧_Γ \∩ \⟦ \T \⟧_Γ \= \{ \}$ follows directly.

      \Case\DArrow:
      \quad $\S \= \S_{1} \→ \S_{2}$
      \quad $\T \= \C_{1}$

      $\⟦ \C_{1} \⟧_Γ \= \{ \c \∈ \C~\|~\(\c, \C_{1}) \∈ \Ψ \}$, $\⟦ \S_{1} \→ \S_{2} \⟧_Γ \= \{ \Λ \}$ and $\⟦ \C_{1} \⟧_Γ \∩ \⟦ \S_{1} \→ \S_{2} \⟧_Γ \= \{ \}$, as required.

      \Case\DAll:
      \quad $\S \= \∀ \X \<: \S_{1}.~ \S_{2}$
      \quad $\T \= \C_{1}$

      $\⟦ \C_{1} \⟧_Γ \= \{ \c \∈ \C~\|~\(\c, \C_{1}) \∈ \Ψ \}$, $\⟦ \∀ \X \<: \S_{1}.~ \S_{2} \⟧_Γ \= \{ \S \}$ and $\⟦ \C_{1} \⟧_Γ \∩ \⟦ \∀ \X \<: \S_{1}.~ \S_{2} \⟧_Γ \= \{ \}$, as required.
    \end{itemize}
  \end{enumerate}
\end{proof}
