\begin{lemma}[Inversion of subtyping]
  \lbl{lem:inversion-of-subtyping}~
  \begin{enumerate}
    \item % 0
    If $\Γ \⊢ \S_s \match \{ \U_i \⇒ \S_i \} \otherwise \S_d \⇌ \T$, then either:
    \begin{enumerate}
      \item
        $\Γ \⊢ \S_s \<: \U_n$,
        $\∀ \m<n.~ \Γ \⊢ \disj(\S_s, \U_m)$
        and $\S_n$ is a match type with $\Γ \⊢ \S_n \⇌ \T$,
      \item
        $\Γ \⊢ \S_s \<: \U_n$,
        $\∀ \m<n.~ \Γ \⊢ \disj(\S_s, \U_m)$
        and $\S_n \= \T$,
      \item
        $\∀ \n.~ \Γ \⊢ \disj(\S_s, \U_n)$
        and $\S_d$ is a match type with $\Γ \⊢ \S_d \⇌ \T$,
      \item
        $\∀ \n.~ \Γ \⊢ \disj(\S_s, \U_n)$
        and $\S_d \= \T$.
    \end{enumerate}

    \item % 1
    If $\Γ \⊢ \S \<: \X$, or $\Γ \⊢ \S \<: \T$ where $\T$ is a match type with $\Γ \⊢ \T \⇌ \X$, then either
    \begin{enumerate}
      \item $\S$ is a match type with $\Γ \⊢ \S \⇌ \Y$, for some $\Y$,
      \item $\S$ is a type variable.
    \end{enumerate}

    \item % 2
    If $\Γ \⊢ \S \<: \T_{1} \→ \T_{2}$, or $\Γ \⊢ \S \<: \T$ where $\T$ is a match type with $\Γ \⊢ \T \⇌ \T_{1} \→ \T_{2}$, then either
    \begin{enumerate}
      \item $\S$ is a match type with $\Γ \⊢ \S \⇌ \S_{1} \→ \S_{2}$, for some $\S_{1}$, $\S_{2}$ such that $\Γ \⊢ \T_{1} \<: \S_{1}$ and $\Γ \⊢ \S_{2} \<: \T_{2}$,
      \item $\S$ is a match type with $\Γ \⊢ \S \⇌ \X$, for some $\X$,
      \item $\S$ is a type variable,
      \item $\S$ has the form $\S_{1} \→ \S_{2}$ with $\Γ \⊢ \T_{1} \<: \S_{1}$ and $\Γ \⊢ \S_{2} \<: \T_{2}$.
    \end{enumerate}

    \item % 3
    If $\Γ \⊢ \S \<: \∀ \X \<: \U_{1}.~ \T_{2}$, or $\Γ \⊢ \S \<: \T$ where $\T$ is a match type with $\Γ \⊢ \T \⇌ \∀ \X \<: \U_{1}.~ \T_{2}$, then either
    \begin{enumerate}
      \item $\S$ is a match type with $\Γ \⊢ \S \⇌ \∀ \X \<: \U_{1}.~ \S_{2}$, for some $\S_{2}$ such that $\Γ, \X \<: \U_{1} \⊢ \S_{2} \<: \T_{2}$,
      \item $\S$ is a match type with $\Γ \⊢ \S \⇌ \X$, for some $\X$,
      \item $\S$ is a type variable,
      \item $\S$ has the form $\∀ \X \<: \U_{1}.~ \S_{2}$ and $\Γ, \X \<: \U_{1} \⊢ \S_{2} \<: \T_{2}$.
    \end{enumerate}
  \end{enumerate}
\end{lemma}

\begin{proof}
  \begin{enumerate}
    \item % 0
    By induction on a derivation on $\Γ \⊢ \S \⇌ \T$.
    \begin{itemize}
      \Case\SMatch1/2:
      \quad $\S \= \S_s \match \{ \U_i \⇒ \S_i \} \otherwise \S_d$
      \quad $\T \= \S_n$
      \\\phantom{\textit{Case}~\SMatch1/2:}
      \quad $\Γ \⊢ \S_s \<: \U_n$
      \quad $\∀ \m<n.~ \Γ \⊢ \disj(\S_s, \U_m)$

      $\S_n \= \T$ and the result follows directly from the premises.

      \Case\SMatch3/4:
      \quad $\S \= \S_s \match \{ \U_i \⇒ \S_i \} \otherwise \S_d$
      \quad $\T \= \S_d$
      \quad $\∀ \n.~ \Γ \⊢ \disj(\S_s, \U_n)$

      $\S_d \= \T$ and the result follows directly from the premises.

      \Case\STrans:
      \quad $\Γ \⊢ \S \⇌ \U$
      \quad $\Γ \⊢ \U \⇌ \T$

      By definition of the $\⇌$ relation, $\Γ \⊢ \U \⇌ \T$ implies that $\U$ is a match type.
      Using the IH on the left premise we get 4 subcases:
      \begin{enumerate}
        \item
          $\Γ \⊢ \S_s \<: \U_n$,
          $\∀ \m<n.~ \Γ \⊢ \disj(\S_s, \U_m)$
          and $\S_n$ is a match type with $\Γ \⊢ \S_n \⇌ \U$:

        Using \STrans we get $\Γ \⊢ \S_n \⇌ \T$, as requiered.

        \item
          $\Γ \⊢ \S_s \<: \U_n$,
          $\∀ \m<n.~ \Γ \⊢ \disj(\S_s, \U_m)$
          and $\S_n \= \U$:

        The result is immediate.

        \item
          $\∀ \n.~ \Γ \⊢ \disj(\S_s, \U_n)$
          and $\S_d$ is a match type with $\Γ \⊢ \S_d \⇌ \U$:

        Using \STrans we get $\Γ \⊢ \S_d \⇌ \T$, as requiered.

        \item
          $\∀ \n.~ \Γ \⊢ \disj(\S_s, \U_n)$
          and $\S_d \= \U$:

        The result is immediate.
      \end{enumerate}
    \end{itemize}

    \item % 1
    By induction on a derivation of $\Γ \⊢ \S \<: \T$.
    \begin{itemize}
      \Case\SRefl:
      \quad $\S \= \T$

      If $\T$ is a type variable then so is $\S$ and the result is immediate.
      If $\T$ is a match type with $\Γ \⊢ \T \⇌ \X$, $\S$ is a match type and $\Γ \⊢ \S \⇌ \Y$, as required.

      \Case\STrans:
      \quad $\Γ \⊢ \S \<: \U$
      \quad $\Γ \⊢ \U \<: \T$

      If $\T$ is a type variable we use the IH on the right premise to get that $\U$ is either a match type with $\Γ \⊢ \U \⇌ \X$ or a type variable.
      In either case, the result follows from using the IH on the left premise.

      If $\T$ is a match type with $\Γ \⊢ \T \⇌ \X$ we can also use the the IH on the right premise to get to the same result.

      \Case\STvar:
      \quad $\S \= \Y$
      \quad $\Y \<: \X \∈ \Γ$

      $\S$ is a type variable and the result is immediate.

      \Case\SMatch1:
      \quad $\S \= \T_s \match \{ \U_i \⇒ \T_i \} \otherwise \T_d$
      \quad $\T \= \T_n$
      \\\phantom{\textit{Case}~\SMatch1:}
      \quad $\Γ \⊢ \T_s \<: \U_n$
      \quad $\∀ \m<n.~ \Γ \⊢ \disj(\T_s, \U_m)$

      If $\T \= \X$, we use \SMatch2 to obtain $\Γ \⊢ \S \⇌ \X$, as required.
      If $\T$ is a match type with $\Γ \⊢ \T \⇌ \X$, we use \SMatch2 to get $\Γ \⊢ \S \⇌ \T$, and \STrans to obtain $\Γ \⊢ \S \⇌ \X$, as required.

      \Case\SMatch2:
      \quad $\S \= \T_n$
      \quad $\T \= \T_s \match \{ \U_i \⇒ \T_i \} \otherwise \T_d$
      \\\phantom{\textit{Case}~\SMatch2:}
      \quad $\Γ \⊢ \T_s \<: \U_n$
      \quad $\∀ \m<n.~ \Γ \⊢ \disj(\T_s, \U_m)$

      In this case the first premise of the statement ($\Γ \⊢ \S \<: \X$) does not apply since $\T$ is a match type.
      Using the 1st part of the lemma on $\Γ \⊢ \T \⇌ \X$, we get 4 subcases:
      \begin{enumerate}
        \item
          $\Γ \⊢ \T_s \<: \U_n$,
          $\∀ \m<n.~ \Γ \⊢ \disj(\T_s, \U_m)$
          and $\T_n$ is a match type with $\Γ \⊢ \T_n \⇌ \X$:

        $\S \= \T_n$ is a match type and $\Γ \⊢ \S \⇌ \X$, as required.

        \item
          $\Γ \⊢ \T_s \<: \U_n$,
          $\∀ \m<n.~ \Γ \⊢ \disj(\T_s, \U_m)$
          and $\T_n \= \X$:

        $\S \= \T_n \= \X$ is a type variable and the result is immediate.

        \item
          $\∀ \m.~ \Γ \⊢ \disj(\T_s, \U_m)$
          and $\T_d$ is a match type with $\Γ \⊢ \T_d \⇌ \X$:

        This case cannot occur since $\Γ \⊢ \disj(\T_s, \U_n)$ would contradict $\Γ \⊢ \T_s \<: \U_n$, by \Cref{lem:disjointness-subtyping-exclusivity}.

        \item
          $\∀ \m.~ \Γ \⊢ \disj(\T_s, \U_m)$
          and $\T_d \= \X$:

        This case cannot occur since $\Γ \⊢ \disj(\T_s, \U_n)$ would contradict $\Γ \⊢ \T_s \<: \U_n$, by \Cref{lem:disjointness-subtyping-exclusivity}.
      \end{enumerate}

      \Case\SMatch3:
      \quad $\S \= \T_s \match \{ \U_i \⇒ \T_i \} \otherwise \T_d$
      \quad $\T \= \T_d$
      \\\phantom{\textit{Case}~\SMatch3:}
      \quad $\∀ \n.~ \Γ \⊢ \disj(\T_s, \U_n)$

      If $\T \= \X$, we use \SMatch4 to obtain $\Γ \⊢ \S \⇌ \X$, as required.
      If $\T$ is a match type with $\Γ \⊢ \T \⇌ \X$, we use \SMatch4 to get $\Γ \⊢ \S \⇌ \T$, and \STrans to obtain $\Γ \⊢ \S \⇌ \X$, as required.

      \Case\SMatch4:
      \quad $\S \= \T_d$
      \quad $\T \= \T_s \match \{ \U_i \⇒ \T_i \} \otherwise \T_d$
      \\\phantom{\textit{Case}~\SMatch3:}
      \quad $\∀ \n.~ \Γ \⊢ \disj(\T_s, \U_n)$

      In this case the first premise of the statement ($\Γ \⊢ \S \<: \X$) does not apply since $\T$ is a match type.
      Using the 1st part of the lemma on $\Γ \⊢ \T \⇌ \X$, we get 4 subcases:
      \begin{enumerate}
        \item
          $\Γ \⊢ \T_s \<: \U_n$,
          $\∀ \m<n.~ \Γ \⊢ \disj(\T_s, \U_m)$
          and $\T_n$ is a match type with $\Γ \⊢ \T_n \⇌ \X$:

        This case cannot occur since $\Γ \⊢ \disj(\T_s, \U_n)$ would contradict $\Γ \⊢ \T_s \<: \U_n$, by \Cref{lem:disjointness-subtyping-exclusivity}.

        \item
          $\Γ \⊢ \T_s \<: \U_n$,
          $\∀ \m<n.~ \Γ \⊢ \disj(\T_s, \U_m)$
          and $\T_n \= \X$:

        This case cannot occur since $\Γ \⊢ \disj(\T_s, \U_n)$ would contradict $\Γ \⊢ \T_s \<: \U_n$, by \Cref{lem:disjointness-subtyping-exclusivity}.

        \item
          $\∀ \m.~ \Γ \⊢ \disj(\T_s, \U_m)$
          and $\T_d$ is a match type with $\Γ \⊢ \T_d \⇌ \X$:

        $\S$ is a match type and $\Γ \⊢ \S \⇌ \X$, as required.

        \item
          $\∀ \m.~ \Γ \⊢ \disj(\T_s, \U_m)$
          and $\T_d \= \X$:

        $\S \= \T_d \= \X$ is a type variable and the result is immediate.
      \end{enumerate}

      \Case\SMatch5:
      \quad $\S \= \S_s \match \{ \U_i \⇒ \S_i \} \otherwise \S_d$
      \quad $\T \= \T_s \match \{ \U_i \⇒ \T_i \} \otherwise \T_d$
      \\\phantom{\textit{Case}~\SMatch5:}
      \quad $\Γ \⊢ \S_s \<: \T_s$
      \quad $\∀ \m.~ \Γ \⊢ \S_m \<: \T_m$
      \quad $\Γ \⊢ \S_d \<: \T_d$

      In this case the first premise of the statement ($\Γ \⊢ \S \<: \X$) does not apply since $\T$ is a match type.
      Using the 1st part of the lemma on $\Γ \⊢ \T \⇌ \X$, we get 4 subcases:
      \begin{enumerate}
        \item
          $\Γ \⊢ \T_s \<: \U_n$,
          $\∀ \m<n.~ \Γ \⊢ \disj(\T_s, \U_m)$
          and $\T_n$ is a match type with $\Γ \⊢ \T_n \⇌ \X$:

        Using \DSub on $\∀ \m.~ \Γ \⊢ \S_m \<: \T_m$ and $\∀ \m<n.~ \Γ \⊢ \disj(\T_s, \U_m)$ we obtain $\∀ \m<n.~ \Γ \⊢ \disj(\S_s, \U_m)$.
        From \STrans we obtain $\Γ \⊢ \S_s \<: \U_n$.
        Using \SMatch1/2 we get $\Γ \⊢ \S \⇌ \S_n$.
        Since $\Γ \⊢ \S_n \<: \T_n$ and $\Γ \⊢ \T_n \⇌ \X$, we use the IH to get that either $\S_n$ is a match type with $\Γ \⊢ \S_n \⇌ \Y$, or $\S_n$ is a type variable.
        If $\S_n$ is a type variable, we already proved $\Γ \⊢ \S \⇌ \S_n$, as required.
        If $\Γ \⊢ \S_n \⇌ \Y$, then using \STrans with $\Γ \⊢ \S \⇌ \S_n$ gives $\Γ \⊢ \S \⇌ \Y$, as required.

        \item
          $\Γ \⊢ \T_s \<: \U_n$,
          $\∀ \m<n.~ \Γ \⊢ \disj(\T_s, \U_m)$
          and $\T_n \= \X$:

        Using \DSub on $\∀ \m.~ \Γ \⊢ \S_m \<: \T_m$ and $\∀ \m<n.~ \Γ \⊢ \disj(\T_s, \U_m)$ we obtain $\∀ \m<n.~ \Γ \⊢ \disj(\S_s, \U_m)$.
        From \STrans we obtain $\Γ \⊢ \S_s \<: \U_n$.
        Using \SMatch1/2 we get $\Γ \⊢ \S \⇌ \S_n$.
        Since $\Γ \⊢ \S_n \<: \X$, we use the IH to get that either $\S_n$ is a match type with $\Γ \⊢ \S_n \⇌ \Y$, or $\S_n$ is a type variable.
        If $\S_n$ is a type variable, we already proved $\Γ \⊢ \S \⇌ \S_n$, as required.
        If $\Γ \⊢ \S_n \⇌ \Y$, then using \STrans with $\Γ \⊢ \S \⇌ \S_n$ gives $\Γ \⊢ \S \⇌ \Y$, as required.

        \item
          $\∀ \m.~ \Γ \⊢ \disj(\T_s, \U_m)$
          and $\T_d$ is a match type with $\Γ \⊢ \T_d \⇌ \X$:

        Using \DSub on $\∀ \m.~ \Γ \⊢ \S_m \<: \T_m$ and $\∀ \m.~ \Γ \⊢ \disj(\T_s, \U_m)$ we obtain $\∀ \m.~ \Γ \⊢ \disj(\S_s, \U_m)$.
        Using \SMatch3/4 we obtain $\Γ \⊢ \S \⇌ \S_d$.
        Since $\Γ \⊢ \S_d \<: \T_d$ and $\Γ \⊢ \T_d \⇌ \X$, we use the IH to get that either $\S_d$ is a match type with $\Γ \⊢ \S_d \⇌ \Y$, or $\S_d$ is a type variable.
        If $\S_d$ is a type variable, we already proved $\Γ \⊢ \S \⇌ \S_d$, as required.
        If $\Γ \⊢ \S_d \⇌ \Y$, then using \STrans with $\Γ \⊢ \S \⇌ \S_d$ gives $\Γ \⊢ \S \⇌ \Y$, as required.

        \item
          $\∀ \m.~ \Γ \⊢ \disj(\T_s, \U_m)$
          and $\T_d \= \X$:

        Using \DSub on $\∀ \m.~ \Γ \⊢ \S_m \<: \T_m$ and $\∀ \m.~ \Γ \⊢ \disj(\T_s, \U_m)$ we obtain $\∀ \m.~ \Γ \⊢ \disj(\S_s, \U_m)$.
        Using \SMatch3/4 we obtain $\Γ \⊢ \S \<: \S_d$.
        Since $\Γ \⊢ \S_d \<: \T_d$ and $\Γ \⊢ \T_d \⇌ \X$, we use the IH to get that either $\S_d$ is a match type with $\Γ \⊢ \S_d \⇌ \Y$, or $\S_d$ is a type variable.
        If $\S_d$ is a type variable, we already proved $\Γ \⊢ \S \⇌ \S_d$, as required.
        If $\Γ \⊢ \S_d \⇌ \Y$, then using \STrans with $\Γ \⊢ \S \⇌ \S_d$ gives $\Γ \⊢ \S \⇌ \Y$, as required.
      \end{enumerate}

      \Case\STop, \SSin, \SArrow, \SAll, \SPsi:

      In those cases, $\T$ is neither a type variable nor a match type and the result is immediate.
    \end{itemize}

    \item % 2
    By induction on a derivation of $\Γ \⊢ \S \<: \T$.
    \begin{itemize}
      \Case\SRefl:
      \quad $\T \= \S$

      If $\T$ is a match type with $\Γ \⊢ \T \⇌ \T_{1} \→ \T_{2}$ and the result follows directly from \SRefl.
      If $\T \= \T_{1} \→ \T_{2}$, the result also follows directly from \SRefl.

      \Case\STrans:
      \quad $\Γ \⊢ \S \<: \U$
      \quad $\Γ \⊢ \U \<: \T$

      Using the IH on the right premise we get 4 subcases:
      \begin{enumerate}
        \item $\U$ is a match type with $\Γ \⊢ \U \⇌ \U_{1} \→ \U_{2}$, for some $\U_{1}$, $\U_{2}$ such that $\Γ \⊢ \T_{1} \<: \U_{1}$ and $\Γ \⊢ \U_{2} \<: \T_{2}$:

        The result follows directly from using the IH on the left premise ($\Γ \⊢ \T_{1} \<: \S_{1}$ is obtained using \STrans with $\Γ \⊢ \T_{1} \<: \U_{1}$ and $\Γ \⊢ \U_{1} \<: \S_{1}$, $\Γ \⊢ \S_{2} \<: \T_{2}$ is obtained analogously).

        \item $\U$ is a match type with $\Γ \⊢ \U \⇌ \X$, for some $\X$:

        Using the 2nd part of the lemma on the left premise leads to the desired result.

        \item $\U$ is a type variable:

        The result follows from using the 2nd part of the lemma on the left premise.

        \item $\U$ has the form $\U_{1} \→ \U_{2}$ with $\Γ \⊢ \T_{1} \<: \U_{1}$ and $\Γ \⊢ \U_{2} \<: \T_{2}$:

        The result follows directly from using the IH on the left premise ($\Γ \⊢ \T_{1} \<: \S_{1}$ is obtained using \STrans with $\Γ \⊢ \T_{1} \<: \U_{1}$ and $\Γ \⊢ \U_{1} \<: \S_{1}$, $\Γ \⊢ \S_{2} \<: \T_{2}$ is obtained analogously).
      \end{enumerate}

      \Case\SMatch1:
      \quad $\S \= \T_s \match \{ \U_i \⇒ \T_i \} \otherwise \T_d$
      \quad $\T \= \T_n$
      \\\phantom{\textit{Case}~\SMatch1:}
      \quad $\Γ \⊢ \T_s \<: \U_n$
      \quad $\∀ \m<n.~ \Γ \⊢ \disj(\T_s, \U_m)$

      If $\T \= \T_{1} \→ \T_{2}$, we use \SMatch2 to obtain $\Γ \⊢ \S \⇌ \T_{1} \→ \T_{2}$, as required.
      If $\T$ is a match type with $\Γ \⊢ \T \⇌ \T_{1} \→ \T_{2}$, we use \SMatch2 to get $\Γ \⊢ \S \⇌ \T$, and \STrans to obtain $\Γ \⊢ \S \⇌ \T_{1} \→ \T_{2}$, and the result follows from \SRefl.

      \Case\SMatch2:
      \quad $\S \= \T_n$
      \quad $\T \= \T_s \match \{ \U_i \⇒ \T_i \} \otherwise \T_d$
      \\\phantom{\textit{Case}~\SMatch2:}
      \quad $\Γ \⊢ \T_s \<: \U_n$
      \quad $\∀ \m<n.~ \Γ \⊢ \disj(\T_s, \U_m)$

      In this case the first premise of the statement ($\Γ \⊢ \S \<: \T_{1} \→ \T_{2}$) does not apply since $\T$ is a match type.
      Using the 1st part of the lemma on $\Γ \⊢ \T \⇌ \T_{1} \→ \T_{2}$, we get 4 subcases:

      \begin{enumerate}
        \item
          $\Γ \⊢ \T_s \<: \U_n$,
          $\∀ \m<n.~ \Γ \⊢ \disj(\T_s, \U_m)$
          and $\T_n$ is a match type with $\Γ \⊢ \T_n \⇌ \T_{1} \→ \T_{2}$:

        $\S \= \T_n$ is a match type and $\Γ \⊢ \S \⇌ \T_{1} \→ \T_{2}$, and the result follows from \SRefl.

        \item
          $\Γ \⊢ \T_s \<: \U_n$,
          $\∀ \m<n.~ \Γ \⊢ \disj(\T_s, \U_m)$
          and $\T_n \= \T_{1} \→ \T_{2}$:

        $\S \= \T_{1} \→ \T_{2}$, $\S_{1} \= \T_{1}$, $\S_{2} \= \T_{2}$, and the result follows from \SRefl.

        \item
          $\∀ \m.~ \Γ \⊢ \disj(\T_s, \U_m)$
          and $\T_d$ is a match type with $\Γ \⊢ \T_d \⇌ \T_{1} \→ \T_{2}$:

        This case cannot occur since $\Γ \⊢ \disj(\T_s, \U_n)$ would contradict $\Γ \⊢ \T_s \<: \U_n$, by \Cref{lem:disjointness-subtyping-exclusivity}.

        \item
          $\∀ \m.~ \Γ \⊢ \disj(\T_s, \U_m)$
          and $\T_d \= \T_{1} \→ \T_{2}$:

        This case cannot occur since $\Γ \⊢ \disj(\T_s, \U_n)$ would contradict $\Γ \⊢ \T_s \<: \U_n$, by \Cref{lem:disjointness-subtyping-exclusivity}.
      \end{enumerate}

      \Case\SMatch3:
      \quad $\S \= \T_s \match \{ \U_i \⇒ \T_i \} \otherwise \T_d$
      \quad $\T \= \T_d$
      \\\phantom{\textit{Case}~\SMatch3:}
      \quad $\∀ \n.~ \Γ \⊢ \disj(\T_s, \U_n)$

      If $\T \= \T_{1} \→ \T_{2}$, we use \SMatch2 to obtain $\Γ \⊢ \S \⇌ \T_{1} \→ \T_{2}$, and the result follows from \SRefl.
      If $\T$ is a match type with $\Γ \⊢ \T \⇌ \T_{1} \→ \T_{2}$, we use \SMatch2 to get $\Γ \⊢ \S \⇌ \T$, and \STrans to obtain $\Γ \⊢ \S \⇌ \T_{1} \→ \T_{2}$, and the result follows from \SRefl.

      \Case\SMatch4:
      \quad $\S \= \T_d$
      \quad $\T \= \T_s \match \{ \U_i \⇒ \T_i \} \otherwise \T_d$
      \\\phantom{\textit{Case}~\SMatch3:}
      \quad $\∀ \n.~ \Γ \⊢ \disj(\T_s, \U_n)$

      In this case the first premise of the statement ($\Γ \⊢ \S \<: \T_{1} \→ \T_{2}$) does not apply since $\T$ is a match type.
      Using the 1st part of the lemma on $\Γ \⊢ \T \⇌ \T_{1} \→ \T_{2}$, we get 4 subcases:

      \begin{enumerate}
        \item
          $\Γ \⊢ \T_s \<: \U_n$,
          $\∀ \m<n.~ \Γ \⊢ \disj(\T_s, \U_m)$
          and $\T_n$ is a match type with $\Γ \⊢ \T_n \⇌ \T_{1} \→ \T_{2}$:

        This case cannot occur since $\Γ \⊢ \disj(\T_s, \U_n)$ would contradict $\Γ \⊢ \T_s \<: \U_n$, by \Cref{lem:disjointness-subtyping-exclusivity}.

        \item
          $\Γ \⊢ \T_s \<: \U_n$,
          $\∀ \m<n.~ \Γ \⊢ \disj(\T_s, \U_m)$
          and $\T_n \= \T_{1} \→ \T_{2}$:

        This case cannot occur since $\Γ \⊢ \disj(\T_s, \U_n)$ would contradict $\Γ \⊢ \T_s \<: \U_n$, by \Cref{lem:disjointness-subtyping-exclusivity}.

        \item
          $\∀ \m.~ \Γ \⊢ \disj(\T_s, \U_m)$
          and $\T_d$ is a match type with $\Γ \⊢ \T_d \⇌ \T_{1} \→ \T_{2}$:

        $\S \= \T_d$ is a match type and $\Γ \⊢ \S \⇌ \T_{1} \→ \T_{2}$, and the result follows from \SRefl.

        \item
          $\∀ \m.~ \Γ \⊢ \disj(\T_s, \U_m)$
          and $\T_d \= \T_{1} \→ \T_{2}$:

        $\S \= \T_{1} \→ \T_{2}$, $\S_{1} \= \T_{1}$, $\S_{2} \= \T_{2}$, and the result follows from \SRefl.
      \end{enumerate}

      \Case\SMatch5:
      \quad $\S \= \S_s \match \{ \U_i \⇒ \S_i \} \otherwise \S_d$
      \quad $\T \= \T_s \match \{ \U_i \⇒ \T_i \} \otherwise \T_d$
      \\\phantom{\textit{Case}~\SMatch5:}
      \quad $\Γ \⊢ \S_s \<: \T_s$
      \quad $\∀ \n.~ \Γ \⊢ \S_n \<: \T_n$
      \quad $\Γ \⊢ \S_d \<: \T_d$

      In this case the first premise of the statement ($\Γ \⊢ \S \<: \T_{1} \→ \T_{2}$) does not apply since $\T$ is a match type.
      Using the 1st part of the lemma on $\Γ \⊢ \T \⇌ \T_{1} \→ \T_{2}$, we get 4 subcases:
      \begin{enumerate}
        \item
          $\Γ \⊢ \T_s \<: \U_n$,
          $\∀ \m<n.~ \Γ \⊢ \disj(\T_s, \U_m)$
          and $\T_n$ is a match type with $\Γ \⊢ \T_n \⇌ \T_{1} \→ \T_{2}$:

        Using \DSub on $\∀ \m.~ \Γ \⊢ \S_m \<: \T_m$ and $\∀ \m<n.~ \Γ \⊢ \disj(\T_s, \U_m)$ we obtain $\∀ \m<n.~ \Γ \⊢ \disj(\S_s, \U_m)$.
        From \STrans, $\Γ \⊢ \S_s \<: \U_n$.
        Using \SMatch1/2 we get $\Γ \⊢ \S \⇌ \S_n$.
        Since $\Γ \⊢ \S_n \<: \T_n$ and $\Γ \⊢ \T_n \⇌ \T_{1} \→ \T_{2}$, the IH gives 4 subsubcases:

        \begin{enumerate}
          \item $\S_n$ is a match type with $\Γ \⊢ \S_n \⇌ \S_{1} \→ \S_{2}$ for some $\S_{1}$, $\S_{2}$ such that $\Γ \⊢ \T_{1} \<: \S_{1}$ and $\Γ \⊢ \S_{2} \<: \T_{2}$:

          The result follows directly from \STrans.

          \item $\S_n$ is a match type with $\Γ \⊢ \S_n \⇌ \X$:

          The result follows directly from \STrans.

          \item $\S_n$ is a type variable:

          We already proved $\Γ \⊢ \S \⇌ \S_n$, as required.

          \item $\S_n$ has the form $\S_{1} \→ \S_{2}$ with $\Γ \⊢ \T_{1} \<: \S_{1}$ and $\Γ \⊢ \S_{2} \<: \T_{2}$:

          We already proved $\Γ \⊢ \S \⇌ \S_n$, as required.
        \end{enumerate}

        \item
          $\Γ \⊢ \T_s \<: \U_n$,
          $\∀ \m<n.~ \Γ \⊢ \disj(\T_s, \U_m)$
          and $\T_n \= \T_{1} \→ \T_{2}$:

        Using \DSub on $\∀ \m.~ \Γ \⊢ \S_m \<: \T_m$ and $\∀ \m<n.~ \Γ \⊢ \disj(\T_s, \U_m)$ we obtain $\∀ \m<n.~ \Γ \⊢ \disj(\S_s, \U_m)$.
        From \STrans, $\Γ \⊢ \S_s \<: \U_n$.
        Using \SMatch1/2 we get $\Γ \⊢ \S \⇌ \S_n$.
        Since $\Γ \⊢ \S_n \<: \T_n$, the IH gives 4 subsubcases:

        \begin{enumerate}
          \item $\S_n$ is a match type with $\Γ \⊢ \S_n \⇌ \S_{1} \→ \S_{2}$ for some $\S_{1}$, $\S_{2}$ such that $\Γ \⊢ \T_{1} \<: \S_{1}$ and $\Γ \⊢ \S_{2} \<: \T_{2}$:

          The result follows directly from \STrans.

          \item $\S_n$ is a match type with $\Γ \⊢ \S_n \⇌ \X$:

          The result follows directly from \STrans.

          \item $\S_n$ is a type variable:

          We already proved $\Γ \⊢ \S \⇌ \S_n$, as required.

          \item $\S_n$ has the form $\S_{1} \→ \S_{2}$ with $\Γ \⊢ \T_{1} \<: \S_{1}$ and $\Γ \⊢ \S_{2} \<: \T_{2}$:

          We already proved $\Γ \⊢ \S \⇌ \S_n$, as required.
        \end{enumerate}

        \item
          $\∀ \n.~ \Γ \⊢ \disj(\T_s, \U_n)$
          and $\T_d$ is a match type with $\Γ \⊢ \T_d \⇌ \T_{1} \→ \T_{2}$:

        Using \DSub on $\∀ \n.~ \Γ \⊢ \S_n \<: \T_n$ and $\∀ \n.~ \Γ \⊢ \disj(\T_s, \U_n)$ we obtain $\∀ \n.~ \Γ \⊢ \disj(\S_s, \U_n)$.
        From \STrans we get $\Γ \⊢ \S_s \<: \C_d$.
        Using \SMatch3/4 we obtain $\Γ \⊢ \S \⇌ \S_d$.
        Since $\Γ \⊢ \S_d \<: \T_d$ and $\Γ \⊢ \T_d \⇌ \T_{1} \→ \T_{2}$, the IH gives 4 subsubcases:

        \begin{enumerate}
          \item $\S_d$ is a match type with $\Γ \⊢ \S_d \⇌ \S_{1} \→ \S_{2}$ for some $\S_{1}$, $\S_{2}$ such that $\Γ \⊢ \T_{1} \<: \S_{1}$ and $\Γ \⊢ \S_{2} \<: \T_{2}$:

          The result follows directly from \STrans.

          \item $\S_d$ is a match type with $\Γ \⊢ \S_d \⇌ \X$:

          The result follows directly from \STrans.

          \item $\S_d$ is a type variable:

          We already proved $\Γ \⊢ \S \⇌ \S_d$, as required.

          \item $\S_d$ has the form $\S_{1} \→ \S_{2}$ with $\Γ \⊢ \T_{1} \<: \S_{1}$ and $\Γ \⊢ \S_{2} \<: \T_{2}$:

          We already proved $\Γ \⊢ \S \⇌ \S_d$, as required.
        \end{enumerate}

        \item
          $\∀ \n.~ \Γ \⊢ \disj(\T_s, \U_n)$
          and $\T_d \= \T_{1} \→ \T_{2}$:

        Using \DSub on $\∀ \n.~ \Γ \⊢ \S_n \<: \T_n$ and $\∀ \n.~ \Γ \⊢ \disj(\T_s, \U_n)$ we obtain $\∀ \n.~ \Γ \⊢ \disj(\S_s, \U_n)$.
        From \STrans we get $\Γ \⊢ \S_s \<: \C_d$.
        Using \SMatch3/4 we obtain $\Γ \⊢ \S \⇌ \S_d$.
        Since $\Γ \⊢ \S_d \<: \T_d$, the IH gives 4 subsubcases:

        \begin{enumerate}
          \item $\S_d$ is a match type with $\Γ \⊢ \S_d \⇌ \S_{1} \→ \S_{2}$ for some $\S_{1}$, $\S_{2}$ such that $\Γ \⊢ \T_{1} \<: \S_{1}$ and $\Γ \⊢ \S_{2} \<: \T_{2}$:

          The result follows directly from \STrans.

          \item $\S_d$ is a match type with $\Γ \⊢ \S_d \⇌ \X$:

          The result follows directly from \STrans.

          \item $\S_d$ is a type variable:

          We already proved $\Γ \⊢ \S \⇌ \S_d$, as required.

          \item $\S_d$ has the form $\S_{1} \→ \S_{2}$ with $\Γ \⊢ \T_{1} \<: \S_{1}$ and $\Γ \⊢ \S_{2} \<: \T_{2}$:

          We already proved $\Γ \⊢ \S \⇌ \S_d$, as required.
        \end{enumerate}
      \end{enumerate}

      \Case\STvar:
      \quad $\S \= \Y$
      \quad $\Y \<: \T \∈ \Γ$

      $\S$ is a type variable and the result is immediate.

      \Case\SArrow:
      \quad $\S \= \S_{1} \→ \S_{2}$
      \quad $\T \= \T_{1} \→ \T_{2}$
      \\\phantom{\textit{Case}~\SArrow:}
      \quad $\Γ \⊢ \T_{1} \<: \S_{1}$
      \quad $\Γ \⊢ \S_{2} \<: \T_{2}$

      $\S$ has the form $\S_{1} \→ \S_{2}$, with $\Γ \⊢ \T_{1} \<: \S_{1}$ and $\Γ \⊢ \S_{2} \<: \T_{2}$, as required.

      \Case\STop, \SSin, \SAll, \SPsi:

      In those cases, $\T$ is neither a type variable nor a function type and the result is immediate.
    \end{itemize}

    \item % 3
    By induction on a derivation of $\Γ \⊢ \S \<: \T$.
    \begin{itemize}
      \Case\SRefl:
      \quad $\T \= \S$

      If $\T$ is a match type with $\Γ \⊢ \T \⇌ \∀ \X \<: \U_{1}.~ \T_{2}$ the result follows directly from \SRefl.
      If $\T \= \∀ \X \<: \U_{1}.~ \T_{2}$, $\S_{2} \= \T_{1}$, and the result also follows from \SRefl.

      \Case\STrans:
      \quad $\Γ \⊢ \S \<: \U$
      \quad $\Γ \⊢ \U \<: \T$

      Using the IH on the right premise we get 4 subcases:
      \begin{enumerate}
        \item $\U$ is a match type with $\Γ \⊢ \U \⇌ \U_{1} \→ \U_{2}$ for some $\U_{2}$ such that $\Γ, \X \<: \U_{1} \⊢ \U_{2} \<: \T_{2}$:

        The result follows directly from using the IH on the left premise ($\Γ, \X \<: \U_{1} \⊢ \S_{2} \<: \T_{2}$ is obtained using \STrans with $\Γ, \X \<: \U_{1} \⊢ \S_{2} \<: \U_{2}$ and $\Γ, \X \<: \U_{1} \⊢ \U_{2} \<: \T_{2}$).

        \item $\U$ is a match type with $\Γ \⊢ \U \⇌ \X$:

        Using the 2nd part of the lemma on the left premise leads to the desired result.

        \item $\U$ is a type variable:

        The result follows from using the 2nd part of the lemma on the left premise.

        \item $\U$ has the form $\∀ \X \<: \U_{1}.~ \U_{2}$ with $\Γ, \X \<: \U_{1} \⊢ \U_{2} \<: \T_{2}$:

        The result follows directly from using the IH on the left premise ($\Γ, \X \<: \U_{1} \⊢ \S_{2} \<: \T_{2}$ is obtained using \STrans with $\Γ, \X \<: \U_{1} \⊢ \S_{2} \<: \U_{2}$ and $\Γ, \X \<: \U_{1} \⊢ \U_{2} \<: \T_{2}$).
      \end{enumerate}

      \Case\SMatch1:
      \quad $\S \= \T_s \match \{ \U_i \⇒ \T_i \} \otherwise \T_d$
      \quad $\T \= \T_n$
      \\\phantom{\textit{Case}~\SMatch1:}
      \quad $\Γ \⊢ \T_s \<: \U_n$
      \quad $\∀ \m<n.~ \Γ \⊢ \disj(\T_s, \U_m)$

      If $\T \= \∀ \X \<: \U_{1}.~ \T_{2}$, we use \SMatch2 to obtain $\Γ \⊢ \S \⇌ \∀ \X \<: \U_{1}.~ \T_{2}$, as required, and the result follows from \SRefl.
      If $\T$ is a match type with $\Γ \⊢ \T \⇌ \∀ \X \<: \U_{1}.~ \T_{2}$, we use \SMatch2 to get $\Γ \⊢ \S \⇌ \T$, and \STrans to obtain $\Γ \⊢ \S \⇌ \∀ \X \<: \U_{1}.~ \T_{2}$, and the result follows from \SRefl.

      \Case\SMatch2:
      \quad $\S \= \T_n$
      \quad $\T \= \T_s \match \{ \U_i \⇒ \T_i \} \otherwise \T_d$
      \\\phantom{\textit{Case}~\SMatch2:}
      \quad $\Γ \⊢ \T_s \<: \U_n$
      \quad $\∀ \m<n.~ \Γ \⊢ \disj(\T_s, \U_m)$

      In this case the first premise of the statement ($\Γ \⊢ \S \<: \(\∀ \X \<: \U_{1}.~ \T_{2})$) does not apply since $\T$ is a match type.
      Using the 1st part of the lemma on $\Γ \⊢ \T \⇌ \∀ \X \<: \U_{1}.~ \T_{2}$, we get 4 subcases:

      \begin{enumerate}
        \item
          $\Γ \⊢ \T_s \<: \U_n$,
          $\∀ \m<n.~ \Γ \⊢ \disj(\T_s, \U_m)$
          and $\T_n$ is a match type with $\Γ \⊢ \T_n \⇌ \∀ \X \<: \U_{1}.~ \T_{2}$:

        $\S \= \T_n$ is a match type and $\Γ \⊢ \S \⇌ \∀ \X \<: \U_{1}.~ \T_{2}$, and the result follows from \SRefl.

        \item
          $\Γ \⊢ \T_s \<: \U_n$,
          $\∀ \m<n.~ \Γ \⊢ \disj(\T_s, \U_m)$
          and $\T_n \= \∀ \X \<: \U_{1}.~ \T_{2}$:

        $\S \= \∀ \X \<: \U_{1}.~ \T_{2}$, $\S_{2} \= \T_{2}$, and the result follows from \SRefl.

        \item
          $\∀ \m.~ \Γ \⊢ \disj(\T_s, \U_m)$
          and $\T_d$ is a match type with $\Γ \⊢ \T_d \⇌ \∀ \X \<: \U_{1}.~ \T_{2}$:

        This case cannot occur since $\Γ \⊢ \disj(\T_s, \U_n)$ would contradict $\Γ \⊢ \T_s \<: \U_n$, by \Cref{lem:disjointness-subtyping-exclusivity}.

        \item
          $\∀ \m.~ \Γ \⊢ \disj(\T_s, \U_m)$
          and $\T_d \= \∀ \X \<: \U_{1}.~ \T_{2}$:

        This case cannot occur since $\Γ \⊢ \disj(\T_s, \U_n)$ would contradict $\Γ \⊢ \T_s \<: \U_n$, by \Cref{lem:disjointness-subtyping-exclusivity}.
      \end{enumerate}

      \Case\SMatch3:
      \quad $\S \= \T_s \match \{ \U_i \⇒ \T_i \} \otherwise \T_d$
      \quad $\T \= \T_d$
      \\\phantom{\textit{Case}~\SMatch3:}
      \quad $\∀ \n.~ \Γ \⊢ \disj(\T_s, \U_n)$

      If $\T \= \∀ \X \<: \U_{1}.~ \T_{2}$, we use \SMatch2 to obtain $\Γ \⊢ \S \⇌ \∀ \X \<: \U_{1}.~ \T_{2}$, and the result follows from \SRefl.
      If $\T$ is a match type with $\Γ \⊢ \T \⇌ \∀ \X \<: \U_{1}.~ \T_{2}$, we use \SMatch2 to get $\Γ \⊢ \S \⇌ \T$, and \STrans to obtain $\Γ \⊢ \S \⇌ \∀ \X \<: \U_{1}.~ \T_{2}$, and the result follows from \SRefl.

      \Case\SMatch4:
      \quad $\S \= \T_d$
      \quad $\T \= \T_s \match \{ \U_i \⇒ \T_i \} \otherwise \T_d$
      \\\phantom{\textit{Case}~\SMatch3:}
      \quad $\∀ \n.~ \Γ \⊢ \disj(\T_s, \U_n)$

      In this case the first premise of the statement ($\Γ \⊢ \S \<: \(\∀ \X \<: \U_{1}.~ \T_{2})$) does not apply since $\T$ is a match type.
      Using the 1st part of the lemma on $\Γ \⊢ \T \⇌ \∀ \X \<: \U_{1}.~ \T_{2}$, we get 4 subcases:

      \begin{enumerate}
        \item
          $\Γ \⊢ \T_s \<: \U_n$,
          $\∀ \m<n.~ \Γ \⊢ \disj(\T_s, \U_m)$
          and $\T_n$ is a match type with $\Γ \⊢ \T_n \⇌ \∀ \X \<: \U_{1}.~ \T_{2}$:

        This case cannot occur since $\Γ \⊢ \disj(\T_s, \U_n)$ would contradict $\Γ \⊢ \T_s \<: \U_n$, by \Cref{lem:disjointness-subtyping-exclusivity}.

        \item
          $\Γ \⊢ \T_s \<: \U_n$,
          $\∀ \m<n.~ \Γ \⊢ \disj(\T_s, \U_m)$
          and $\T_n \= \∀ \X \<: \U_{1}.~ \T_{2}$:

        This case cannot occur since $\Γ \⊢ \disj(\T_s, \U_n)$ would contradict $\Γ \⊢ \T_s \<: \U_n$, by \Cref{lem:disjointness-subtyping-exclusivity}.

        \item
          $\∀ \m.~ \Γ \⊢ \disj(\T_s, \U_m)$
          and $\T_d$ is a match type with $\Γ \⊢ \T_d \⇌ \∀ \X \<: \U_{1}.~ \T_{2}$:

        $\S \= \T_d$ is a match type and $\Γ \⊢ \S \⇌ \∀ \X \<: \U_{1}.~ \T_{2}$, and the result follows from \SRefl.

        \item
          $\∀ \m.~ \Γ \⊢ \disj(\T_s, \U_m)$
          and $\T_d \= \∀ \X \<: \U_{1}.~ \T_{2}$:

        $\S \= \∀ \X \<: \U_{1}.~ \T_{2}$, $\S_{2} \= \T_{2}$, and the result follows from \SRefl.
      \end{enumerate}

      \Case\SMatch5:
      \quad $\S \= \S_s \match \{ \U_i \⇒ \S_i \} \otherwise \S_d$
      \quad $\T \= \T_s \match \{ \U_i \⇒ \T_i \} \otherwise \T_d$
      \\\phantom{\textit{Case}~\SMatch5:}
      \quad $\Γ \⊢ \S_s \<: \T_s$
      \quad $\∀ \n.~ \Γ \⊢ \S_n \<: \T_n$
      \quad $\Γ \⊢ \S_d \<: \T_d$

      In this case the first premise of the statement ($\Γ \⊢ \S \<: \(\∀ \X \<: \U_{1}.~ \T_{2})$) does not apply since $\T$ is a match type.
      Using the 1st part of the lemma on $\Γ \⊢ \T \⇌ \∀ \X \<: \U_{1}.~ \T_{2}$, we get 4 subcases:

      \begin{enumerate}
        \item
          $\Γ \⊢ \T_s \<: \U_n$,
          $\∀ \m<n.~ \Γ \⊢ \disj(\T_s, \U_m)$
          and $\T_n$ is a match type with $\Γ \⊢ \T_n \⇌ \∀ \X \<: \U_{1}.~ \T_{2}$:

        Using \DSub on $\∀ \m.~ \Γ \⊢ \S_m \<: \T_m$ and $\∀ \m<n.~ \Γ \⊢ \disj(\T_s, \U_m)$ we get $\∀ \m<n.~ \Γ \⊢ \disj(\S_s, \U_m)$.
        From \STrans we get $\Γ \⊢ \S_s \<: \U_n$.
        Using \SMatch1/2 we get $\Γ \⊢ \S \⇌ \S_n$.
        Since $\Γ \⊢ \S_n \<: \T_n$ and $\Γ \⊢ \T_n \⇌ \∀ \X \<: \U_{1}.~ \T_{2}$, the IH gives 4 subsubcases:

        \begin{enumerate}
          \item $\S_n$ is a match type with $\Γ \⊢ \S_n \⇌ \∀ \X \<: \U_{1}.~ \S_{2}$ for some $\S_{2}$ such that $\Γ, \X \<: \U_{1} \⊢ \S_{2} \<: \T_{2}$:

          The result follows directly from \STrans.

          \item $\S_n$ is a match type with $\Γ \⊢ \S_n \⇌ \X$:

          The result follows directly from \STrans.

          \item $\S_n$ is a type variable:

          We already proved $\Γ \⊢ \S \⇌ \S_n$, as required.

          \item $\S_n$ has the form $\∀ \X \<: \U_{1}.~ \S_{2}$ with $\Γ, \X \<: \U_{1} \⊢ \S_{2} \<: \T_{2}$:

          We already proved $\Γ \⊢ \S \⇌ \S_n$, as required.
        \end{enumerate}

        \item
          $\Γ \⊢ \T_s \<: \U_n$,
          $\∀ \m<n.~ \Γ \⊢ \disj(\T_s, \U_m)$
          and $\T_n \= \∀ \X \<: \U_{1}.~ \T_{2}$:

        Using \DSub on $\∀ \m.~ \Γ \⊢ \S_m \<: \T_m$ and $\∀ \m<n.~ \Γ \⊢ \disj(\T_s, \U_m)$ we get $\∀ \m<n.~ \Γ \⊢ \disj(\S_s, \U_m)$.
        From \STrans we get $\Γ \⊢ \S_s \<: \U_n$.
        Using \SMatch1/2 we get $\Γ \⊢ \S \⇌ \S_n$.
        Since $\Γ \⊢ \S_n \<: \T_n$, the IH, the IH gives 4 subsubcases:

        \begin{enumerate}
          \item $\S_n$ is a match type with $\Γ \⊢ \S_n \⇌ \∀ \X \<: \U_{1}.~ \S_{2}$ for some $\S_{2}$ such that $\Γ, \X \<: \U_{1} \⊢ \S_{2} \<: \T_{2}$:

          The result follows directly from \STrans.

          \item $\S_n$ is a match type with $\Γ \⊢ \S_n \⇌ \X$:

          The result follows directly from \STrans.

          \item $\S_n$ is a type variable:

          We already proved $\Γ \⊢ \S \⇌ \S_n$, as required.

          \item $\S_n$ has the form $\∀ \X \<: \U_{1}.~ \S_{2}$ with $\Γ, \X \<: \U_{1} \⊢ \S_{2} \<: \T_{2}$:

          We already proved $\Γ \⊢ \S \⇌ \S_n$, as required.
        \end{enumerate}

        \item
          $\∀ \m.~ \Γ \⊢ \disj(\T_s, \U_m)$
          and $\T_d$ is a match type with $\Γ \⊢ \T_d \⇌ \∀ \X \<: \U_{1}.~ \T_{2}$:

        Using \DSub on $\∀ \m.~ \Γ \⊢ \S_m \<: \T_m$ and $\∀ \m.~ \Γ \⊢ \disj(\T_s, \U_m)$ we obtain $\∀ \m.~ \Γ \⊢ \disj(\S_s, \U_m)$.
        From \STrans we get $\Γ \⊢ \S_s \<: \C_d$.
        Using \SMatch3/4 we obtain $\Γ \⊢ \S \⇌ \S_d$.
        Since $\Γ \⊢ \S_d \<: \T_d$ and $\Γ \⊢ \T_d \⇌ \∀ \X \<: \U_{1}.~ \T_{2}$, the IH gives 4 subsubcases:

        \begin{enumerate}
          \item $\S_d$ is a match type with $\Γ \⊢ \S_d \⇌ \∀ \X \<: \U_{1}.~ \S_{2}$ for some $\S_{2}$ such that $\Γ, \X \<: \U_{1} \⊢ \S_{2} \<: \T_{2}$::

          The result follows directly from \STrans.

          \item $\S_d$ is a match type with $\Γ \⊢ \S_d \⇌ \X$:

          The result follows directly from \STrans.

          \item $\S_d$ is a type variable:

          We already proved $\Γ \⊢ \S \⇌ \S_d$, as required.

          \item $\S_d$ has the form $\∀ \X \<: \U_{1}.~ \S_{2}$ with $\Γ, \X \<: \U_{1} \⊢ \S_{2} \<: \T_{2}$:

          We already proved $\Γ \⊢ \S \⇌ \S_d$, as required.
        \end{enumerate}

        \item
          $\∀ \m.~ \Γ \⊢ \disj(\T_s, \U_m)$
          and $\T_d \= \∀ \X \<: \U_{1}.~ \T_{2}$:

        Using \DSub on $\∀ \m.~ \Γ \⊢ \S_m \<: \T_m$ and $\∀ \m.~ \Γ \⊢ \disj(\T_s, \U_m)$ we obtain $\∀ \m.~ \Γ \⊢ \disj(\S_s, \U_m)$.
        From \STrans we get $\Γ \⊢ \S_s \<: \C_d$.
        Using \SMatch3/4 we obtain $\Γ \⊢ \S \⇌ \S_d$.
        Since $\Γ \⊢ \S_d \<: \T_d$, the IH gives 4 subsubcases:

        \begin{enumerate}
          \item $\S_d$ is a match type with $\Γ \⊢ \S_d \⇌ \∀ \X \<: \U_{1}.~ \S_{2}$ for some $\S_{2}$ such that $\Γ, \X \<: \U_{1} \⊢ \S_{2} \<: \T_{2}$::

          The result follows directly from \STrans.

          \item $\S_d$ is a match type with $\Γ \⊢ \S_d \⇌ \X$:

          The result follows directly from \STrans.

          \item $\S_d$ is a type variable:

          We already proved $\Γ \⊢ \S \⇌ \S_d$, as required.

          \item $\S_d$ has the form $\∀ \X \<: \U_{1}.~ \S_{2}$ with $\Γ, \X \<: \U_{1} \⊢ \S_{2} \<: \T_{2}$:

          We already proved $\Γ \⊢ \S \⇌ \S_d$, as required.
        \end{enumerate}
      \end{enumerate}

      \Case\STvar:
      \quad $\S \= \Y$
      \quad $\Y \<: \T \∈ \Γ$

      $\S$ is a type variable and the result is immediate.

      \Case\SAll:
      \quad $\S \= \∀ \X \<: \U_{1}.~ \S_{2}$
      \quad $\T \= \∀ \X \<: \U_{1}.~ \T_{2}$
      \\\phantom{\textit{Case}~\SArrow:}
      \quad $\Γ, \X \<: \U_{1} \⊢ \S_{2} \<: \T_{2}$

      $\S$ has the form $\∀ \X \<: \U_{1}.~ \S_{2}$, with $\Γ, \X \<: \U_{1} \⊢ \S_{2} \<: \T_{2}$, as required.

      \Case\STop, \SSin, \SArrow, \SPsi:

      In those cases, $\T$ is neither a type variable nor a universal type and the result is immediate.
    \end{itemize}
  \end{enumerate}
\end{proof}
