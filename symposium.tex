\section{Introduction}

The Scala standard library includes a package to manipulate regular expressions.
This package is based on Java's implementation of regular expression, and thus benefits from the high performance of the JVM's regex engine.
The Scala implementation innovates with its presentation: it uses the power and flexibility of Scala's pattern matching to offer a syntax that is both elegant and concise.
The package's documentation starts with the following example:

% val date = new Regex("(\\d{4})-(\\d{2})-(\\d{2})")
% "2004-01-20" match {
%   case date(year, month, day) =>
%     s"$year was a good year for PLs."
% }

\noindent
The extractor pattern, |case date(y,m,d) =>|, replaces the need for manually indexing into the regular expression's capture groups, and shields its users from off-by-one error.

While the syntax of this example would undoubtedly make some non-Scala programmers envious, its type safety leaves much to be desired.
First of all, the \emph{number} of variable bindings in the extractor is entirely opaque to Scala's type system, and left to the discretion of the programmer.
Furthermore, the values coming out of capture groups can be null (when using optional captures) and thus require an additional layer of validation.
Those shortcomings might appear benign when looking at small examples, but can easily turn into bugs when dealing with a large-scale code base.

In this chapter, we propose a new design for Scala's regular expression library which provides both type-safe and null-safe capture group extraction.
Our design makes extensive use of match types, Scala~3 new feature for type-level programming, to statically analyze regular expressions during type checking and generate type-specialized extractors.
We build our interface to mimic Scala's original regular expression API so that Scala programmers can use it as a drop-in replacement while still enjoying additional safety.

Our contribution are as follows:

\begin{itemize}
  \item ...
  \item ...
  \item ...
  \item ...
\end{itemize}

\section{Architecture}

In this section...

\section{Type-Level}

In this section...

\section{Term-Level}

In this section...

\section{Evaluation}

In this section...
